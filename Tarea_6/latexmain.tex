\documentclass[12pt]{article}
\usepackage[spanish]{babel}
\usepackage[utf8]{inputenc}
\usepackage{amsmath}
\usepackage{graphicx}
\usepackage{float}
\usepackage{geometry}
\usepackage{setspace}
\usepackage{array}     
\usepackage{xcolor}    
\usepackage{hyperref}



\geometry{a4paper, margin=2.5cm}
\onehalfspacing

\title{Pronóstico de Variables Deportivas Utilizando Modelos Supervisados}
\author{Sergio Cortés Cepeda\\matricula:1731225\\
Universidad Autónoma de Nuevo León, Facultad de Ciencias Físico Matemáticas}
\date{Noviembre 2025}

\begin{document}
\maketitle
\begin{abstract}
Este estudio aplica algoritmos supervisados para predecir dos variables deportivas obtenidas desde datos reales del reloj Garmin: las calorías quemadas y el ritmo promedio de carrera. Se emplearon tres modelos de regresión \textbf{—Regresión Lineal, Random Forest y KNN—} evaluados mediante métricas estándar como \textbf{MAE, MSE y RMSE}. Los resultados muestran que la \textbf{Regresión Lineal} es superior para pronosticar calorías, mientras que \textbf{ Random Forest} obtiene el mejor desempeño al predecir el ritmo promedio.
\end{abstract}

\section{Introducción}

Los modelos supervisados permiten estimar una variable objetivo a partir de un conjunto de predictores.Estos algoritmos pueden utilizarse para estimar rendimiento, esfuerzo y gasto energético. El propósito de esta tarea es construir modelos capaces de predecir:

\begin{itemize}
    \item Calorías quemadas por sesión.
    \item Ritmo promedio (segundos por kilómetro).
\end{itemize}

Para ello se analizaron los datos reales provenientes del reloj Garmin, que describen múltiples sesiones de mis  entrenamientos, incluyendo métricas fisiológicas y de desempeño.

\clearpage
\section{Metodología}

\subsection{Descripción del conjunto de datos}

El conjunto de datos contiene información de cada entrenamiento:

\begin{itemize}
    \item Distancia recorrida (\texttt{distancia\_km})
    \item Tiempo total en segundos (\texttt{Tiempo\_Total})
    \item Calorías (\texttt{calorias})
    \item Frecuencia cardiaca media y máxima (\texttt{fc\_media}, \texttt{fc\_max})
    \item Cadencia media (\texttt{cadencia\_media})
    \item Zancada promedio (\texttt{zancada\_m})
    \item Ritmo promedio (\texttt{ritmo\_medio})
\end{itemize}

\subsection{Variables utilizadas}

Variables predictoras (\textbf{X}):

\begin{itemize}
    \item Distancia
    \item Tiempo total
    \item Frecuencia cardiaca media
    \item Frecuencia cardiaca máxima
    \item Cadencia
    \item Zancada
\end{itemize}

Variables objetivo (\textbf{y}):

\begin{itemize}
    \item Calorías
    \item Ritmo promedio
\end{itemize}

\clearpage
\section{Modelos Supervisados}

Se emplearon tres algoritmos:

\subsection{Regresión Lineal}
Modelo paramétrico basado en combinación lineal de predictores. Es interpretable y funciona bien cuando la relación aproximada es lineal.

\subsection{Random Forest Regressor}
Modelo de ensamble basado en múltiples árboles. Captura relaciones no lineales y es robusto a valores atípicos.

\subsection{KNN Regressor}
Predice valores en función de los \emph{k} vecinos más cercanos. Es sensible al escalado y puede fallar si los datos tienen alta varianza.

\section{Métricas Utilizadas}

\subsection{MAE}
\[
MAE = \frac{1}{n} \sum |y_i - \hat{y}_i|
\]

\subsection{MSE}
\[
MSE = \frac{1}{n} \sum (y_i - \hat{y}_i)^2
\]

\subsection{RMSE}
\[
RMSE = \sqrt{MSE}
\]

Estas métricas se seleccionaron por ser las más comunes en regresión y por permitir evaluar precisión (MAE), penalización de errores grandes (MSE) y magnitud global del error (RMSE).

\section{Resultados}

\subsection{Predicción de Calorías}

\begin{table}[H]
\centering
\caption{Resultados para la predicción de calorías.}
\begin{tabular}{lccc}
\hline
\textbf{Modelo} & \textbf{MAE} & \textbf{MSE} & \textbf{RMSE} \\
\hline
Regresión Lineal & 17.16 & 519.70 & 22.80 \\
Random Forest    & 25.74 & 1062.52 & 32.60 \\
KNN              & 36.88 & 1888.47 & 43.46 \\
\hline
\end{tabular}
\end{table}

\begin{figure}[H]
\centering
\includegraphics[width=0.75\textwidth]{HistModeloCalorias.png}
\caption{{Histograma de errores absolutos — Calorías (Regresión Lineal).}}
\end{figure}

\clearpage
\subsection{Predicción de Ritmo Promedio}

\begin{table}[H]
\centering
\caption{Resultados para la predicción de ritmo promedio.}
\begin{tabular}{lccc}
\hline
\textbf{Modelo} & \textbf{MAE} & \textbf{MSE} & \textbf{RMSE} \\
\hline
Regresión Lineal & 38.09 & 2174.02 & 46.62 \\
Random Forest    & 30.80 & 1543.58 & 39.29 \\
KNN              & 80.52 & 20042.28 & 141.57 \\
\hline
\end{tabular}
\end{table}

\begin{figure}[H]
\centering
\includegraphics[width=0.75\textwidth]{HistModeloRitmoPromedio.png}
\caption{Histograma de errores absolutos — Ritmo Promedio (Random Forest).}
\end{figure}

\section{Discusión}

Los resultados muestran que:

\begin{itemize}
    \item La \textbf{Regresión Lineal} funciona mejor para predecir calorías debido a su relación lineal con distancia, tiempo y frecuencia cardiaca.
    \item El \textbf{Random Forest} es superior prediciendo ritmo promedio, lo cual indica relaciones no lineales entre cadencia, FC y zancada.
    \item \textbf{KNN} obtuvo el peor desempeño en ambas tareas, probablemente por su sensibilidad al ruido y al tamaño reducido del dataset.
\end{itemize}

Los histogramas muestran que existen valores atípicos que afectan el MSE y RMSE, especialmente en el ritmo promedio.

\section{Conclusiones}

\begin{itemize}
    \item Los modelos supervisados permiten realizar pronósticos útiles sobre rendimiento usando datos reales del garmin.
    \item La Regresión Lineal es la mejor opción para estimar calorías quemadas.
    \item Random Forest ofrece la predicción más precisa para el ritmo promedio.
    \item Para mejorar resultados futuros, sería útil incorporar variables externas como temperatura, tipo de superficie o fatiga acumulada.
\end{itemize}


\begin{thebibliography}{9}

\bibitem{gfg-linear-regression}
GeeksforGeeks. (2024). \textit{ML | Linear Regression}. Recuperado de https://www.geeksforgeeks.org/machine-learning/ml-linear-regression/

\bibitem{datacamp-rf}
DataCamp. (2023). \textit{Random Forests in Python}. Recuperado de https://www.datacamp.com/tutorial/random-forests-classifier-python

\bibitem{proclus-metrics}
Proclus Academy. (2023). \textit{Regression Metrics You Must Know}. Recuperado de https://proclusacademy.com/blog/explainer/regression-metrics-you-must-know/

\end{thebibliography}


\end{document}
